
% RTP Notes
%
% H. Motteler
%

\documentclass[12pt]{article}

\usepackage{graphicx}

\DeclareGraphicsExtensions{.pdf}  %for .pdf version
%\DeclareGraphicsExtensions{.eps} %for .ps version

\newif\iftth

\iftth
\begin{html}
<title>RTP Format Specification</title>
<BODY BGCOLOR="#FFFFFF">
\end{html}
\fi

\title{{\bf RTP Implementation Notes}}

\author{Howard~E.~Motteler}

\date{\today\vspace{1cm}}

\begin{document}

\maketitle

\begin{abstract}

This is a collection and catch-all for stuff that didn't belong in
the user's guide or README; it is mostly a discussion of both HDF~4
and more general RTP implementation issues.

\end{abstract}

% \newpage
\bigskip\bigskip


\section{Introduction}

While the Fortran RTP interface is (hopefully) relatively
user-friendly, the Fortran implementation is complicated in part
because it tries to be sensible about variable field sets and field
sizes; this is convenient for the user, but is the sort of thing
that is much easier from a high level language like Matlab, than
from Fortran or~C.  A simpler implementation might map Fortran
structures directly to the HDF vdatas buffer, but this would require
recompilation for every format or field variation, resulting in a
succession of incompatible file formats, and would need to include
space for radiances and observations even when they weren't used.


\section{The Fortran/C API}

The Fortran/C implementation matches up field subsets from the API's
static reord structures and the HDF 4 vdata field sets in a
relatively efficient way.  It builds a common field subset and then
builds a set of header and profile field pointers and lengths used
for subsequent buffer copying.  Since more than one RTP file might
be open at a time, more than one set of pointers must be kept track
of, thus the RTP channels are needed.

The Fortran interface procedures are mainly just wrappers for
lower-level C routines, except for rtpopen.f, which moves attribute
data between static and dynamic structures.

Note that when changing or adding profile or header fields or field
sizes, the C structures and ascii field lists must be kept in exact
correspondence; also the C and Fortran structures and parameters
must match exactly.

rtpwrite1 starts with field lists for the static Fortran header and
profile structures, and uses values in the header size fields (nrho,
nemis, ngas, mlevs, nchan, and pfields) to build new, generally more
compact, field lists for the header and profile vdata buffers.  The
structure and vdata buffer field lists are matched up in pvmatch,
which also generates buffer pointers.  These pointers, the vdata
buffers, and some related info are saved in the RTP channel records.
rtpwrite1 then writes the header data, and the header and profile
attributes, and returns an RTP channel on which the profiles may
be written by rtpwrite2.

The "pv*.c" routines are fairly generic, and implement C-structure
interface to HDF vdatas.  The rtp*.c (and pvmatch.c) routines do the
field matching, buffer mapping and other RTP-specific stuff.  By
changing just the pv layer, it should be possible to move the
existing RTP API to a different underlying representation; for
example RTP might be implemented directly as binary files, writing
flists to the file to keep track of things, or might be implemented
with HDF~5.


\section{General HDF~4 Implementation Issues}

HDF is a plausible choice for representing atmospheric profiles;
it is portable, and can be used in such a way that the data files
are largely self-documenting.  Test implementations of RTP were
done as HDF~4 ``vgroups'' of SDS's with one group per profile, and
as HDF~4 ``vdatas'', with one vdata for the header, and one for the
profile array.  (The HDF~4 SDS format is the basis of the EOS swath
format, and the Vdata format the basis of EOS point; the RTP vdata
setup is similar to an EOS point file with two EOS-point
``levels'', one for the header and one for the profile data.)

The first implementation, based on vgroups of SDSs, was more
flexible, as different profiles can have different level sets, with
no fill values needed, and new fields can be added--e.g., radiances
to a profile set--as proper updates, without having to rewrite the
whole file.  Unfortunately, limitations of HDF~4--both default
compile-time parameters, and also issues of efficiency--restricted
the total number of profiles representable in this way, in a single
file, to on the order of 500.

The second implementation, based on HDF~4 vdatas, allows for much
larger files, up to 2~Gbytes with no real limit on the number of
profiles, up to the overall size limit, with the restriction that at
least within a particular file, the field set and field sizes are
fixed, though these can both vary between files.  Appending profiles
with the same field set works fine with this representation, but
adding new fields requires reading, updating, and then rewriting the
file.  While such limitations are irritating, overall, for HDF4 this
is a more practical approach than using vgroups.

A third possible HDF4 implementation, that I didn't actually try,
would be to take an SDS (i.e. a simple array) based implementation
for a single profile and extend the last (or first, for the
row-oriented) dimension of all the arrays with a profile index.  A
significant limitation of this approach is that only one of the
arrays/fields can be extended once the file is created, and it was
not pursued further.

a language-independent representation of proper structures and
arrays of structures would be preferable to any of the HDF~4
schemes; HDF~5 appears to provide something like this, but was not
available in time for this work.

\end{document}

